\documentclass[addpoints,11pt]{exam}
\usepackage[top=0.5in, bottom=0.5in, left=0.5in, right=0.5in]{geometry}
\usepackage[utf8]{inputenc}
\usepackage{listings}
\usepackage{color,graphicx}
\usepackage{multicol}
\usepackage{MnSymbol}



\definecolor{codegreen}{rgb}{0,0,0}
\definecolor{codegray}{rgb}{0,0,0}
\definecolor{codepurple}{rgb}{0,0,0}
\definecolor{backcolour}{rgb}{1,1,1}

\lstdefinestyle{mystyle}{
backgroundcolor=\color{backcolour},   
commentstyle=\color{codegreen},
keywordstyle=\color{black},
numberstyle=\tiny\color{codegray},
stringstyle=\color{codepurple},
basicstyle=\footnotesize,
breakatwhitespace=false,         
breaklines=true,                 
captionpos=b,                    
keepspaces=true,                 
numbers=left,                    
numbersep=5pt,                  
showspaces=false,                
showstringspaces=false,
showtabs=false,                  
tabsize=2
}

\lstset{style=mystyle}

\pagestyle{empty}

\begin{document}
\boxedpoints
\pointname{~punti}
\begin{center}
\fbox{\fbox{\parbox{5.5in}{\centering
			Prova scritta Programmazione Procedurale}}}
\end{center}

\vspace{5mm}

\noindent\makebox[\textwidth]{Nome e Cognome: \rule{8cm}{.1pt} \hspace{1cm} Matricola:  \rule{5cm}{.1pt}}

%\vspace{5mm}
%\makebox[\textwidth]{Matricola:\enspace\hrulefill}


\begin{questions} 

\question[1]
Evidenziare le conversioni di tipo \underline{implicite} e scrivere cosa viene stampato a video alla fine, che tipo assume alla fine la variabile \emph{sum} e quanto valgono le variabili \emph{x} e \emph{y} alla fine dell'esecuzione del codice. 


\begin{minipage}[t]{0.5\linewidth}
	\begin{lstlisting}[language=C]
	int  x = 13;
	char y = 'C'; /* ASCII value is 67 */
	float sum;
	sum = x + y;
	printf("sum = %f", sum );
	\end{lstlisting}
\end{minipage}
\begin{minipage}[t]{0.5\linewidth}
	\makeemptybox{50pt}
\end{minipage}



\question[1]
Rappresentare i numeri $1$, $-1$ in complemento a due (8bit) 

\begin{minipage}[t]{0.5\linewidth}
	
\end{minipage}
\begin{minipage}[t]{0.5\linewidth}
	\makeemptybox{70pt}
\end{minipage}



\question[2]
Calcolare la seguente somma in complemento a due e scrivere il risultato sia in complemento a due che in base decimale

\begin{minipage}[t]{0.5\linewidth}
	\begin{lstlisting}[language=C]
	00000101 +
	11110101\end{lstlisting}
\end{minipage}
\begin{minipage}[t]{0.5\linewidth}
	\makeemptybox{50pt}
\end{minipage}



\question[1]
Scrivere una funzione che dati in \emph{input} due caratteri qualsiasi, stampi a video il carattere e il relativo valore nella codifica ASCII, con valore ASCII più grande (in caso di uguaglianza si stampi il primo). \\
Es. \emph{input}: ("$a$", "$h$"), i loro valori nella codifica ASCII sono: 97 e 104, quindi l' \emph{output} dovrà essere del tipo: "Il valore ASCII del carattere $h$ è $104$".
\makeemptybox{100pt}



\question[3]
Scrivere una funzione che dati in \emph{input} due numeri interi, calcoli il \emph{massimo comune divisore} di questi.
\makeemptybox{120pt}



\question[3]
Scrivere una funzione che richieda di inserire un numero da tastiera e verifichi se questo è \emph{palindromo}, cioè che si legge uguale sia da sinistra verso destra, che da destra verso sinistra (es. 909). Se la condizione è verificata si ritorni il valore intero $1$, altrimenti $0$.
\makeemptybox{150pt}



\question[3]
Scrivere un programma che dato un array di interi già definito e la sua dimensione (\emph{input}: int $arr[]$, int $arr\_size$), rimuova i duplicati all'interno di questo e stampi i suoi elementi
\makeemptybox{150pt}



\question[2]
Scrivere una funzione cha dato in input un numero intero, stampi a video un triangolo composto da sequenze di numeri incrementali fino al massimo di n e viceversa. Cioè: \\
\begin{minipage}[t]{0.5\linewidth}
	\begin{lstlisting}[language=C]
	Input: 5        Input:3
	1               1
	12              12
	123             123
	1234            12
	12345           1
	1234
	123
	12
	1\end{lstlisting}
\end{minipage}
\makeemptybox{150pt}



\question[2]
Scrivere una funzione \underline{iterativa} che, dato in \emph{input} un numero intero, stampi a video il suo valore nella successione di fibonacci

\makeemptybox{150pt}



\question[2]
Scrivere una funzione \underline{ricorsiva} che dato in \emph{input} un numero intero, stampi a video il suo valore nella successione di fibonacci

\makeemptybox{100}



\question [1]
Scrivere una funzione che, dato in \emph{input} un puntatore (di tipo intero) ed un numero intero, calcoli il fattoriale di questo numero e lo salvi nell'area di memoria indirizzata dal puntatore
\makeemptybox{150pt}



\question[1] 
Definire una struttura per una \emph{linked list} contenente un valore intero e un puntatore al prossimo nodo.
\makeemptybox{100pt}



\question[2] 
Scrivere una funzione che, dato in \emph{input} un puntatore alla testa di una linked list (\emph{*head}), iteri e stampi tutti gli elementi della \emph{linked list} sopra definita
\makeemptybox{120pt}



\question[2]
Allocare dinamicamente un'area di memoria 10x10 di tipo double

\makeemptybox{100pt}



\question[2]
Scrivere cosa stampa a video il seguente codice ed indicare i valori di tipo intero, contenuti all'interno dell'area di memoria indirizzata dal puntatore \emph{ptr}. Inoltre scrivere una porzione di codice che permetta la stampa di tutti gli elementi contenuti nell'area di memoria indirizzata dal puntatore \emph{ptr}.

\begin{minipage}[t]{0.5\linewidth}
	\begin{lstlisting}[language=C]
	#include <stdio.h>
	#include <stdlib.h>

	int main()
	{
		int *ptr;
		ptr = (int *) calloc(5, sizeof(int));
		
		for (int i = 1; i < 5; i++){
			*(ptr+i) = i;
		}
		
		printf("%d", *ptr);
		return 0;
	}
	\end{lstlisting}
\end{minipage}
\begin{minipage}[t]{0.5\linewidth}
	\makeemptybox{150pt}
\end{minipage}




\end{questions}

\end{document}