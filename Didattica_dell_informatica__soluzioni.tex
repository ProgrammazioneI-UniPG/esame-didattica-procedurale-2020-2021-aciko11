\documentclass[addpoints,11pt]{exam}
\usepackage[top=0.5in, bottom=0.5in, left=0.5in, right=0.5in]{geometry}
\usepackage[utf8]{inputenc}
\usepackage{listings}
\usepackage{color,graphicx}
\usepackage{multicol}
\usepackage{MnSymbol}



\definecolor{codegreen}{rgb}{0,0,0}
\definecolor{codegray}{rgb}{0,0,0}
\definecolor{codepurple}{rgb}{0,0,0}
\definecolor{backcolour}{rgb}{1,1,1}

\lstdefinestyle{mystyle}{
backgroundcolor=\color{backcolour},   
commentstyle=\color{codegreen},
keywordstyle=\color{black},
numberstyle=\tiny\color{codegray},
stringstyle=\color{codepurple},
basicstyle=\footnotesize,
breakatwhitespace=false,         
breaklines=true,                 
captionpos=b,                    
keepspaces=true,                 
numbers=left,                    
numbersep=5pt,                  
showspaces=false,                
showstringspaces=false,
showtabs=false,                  
tabsize=2
}

\lstset{style=mystyle}

\pagestyle{empty}

\begin{document}
\boxedpoints
\pointname{~punti}
\begin{center}
\fbox{\fbox{\parbox{5.5in}{\centering
			Prova scritta Programmazione Procedurale}}}
\end{center}

\vspace{5mm}

\noindent\makebox[\textwidth]{Nome e Cognome: \rule{8cm}{.1pt} \hspace{1cm} Matricola:  \rule{5cm}{.1pt}}

%\vspace{5mm}
%\makebox[\textwidth]{Matricola:\enspace\hrulefill}


\begin{questions} 

\question[1]
Evidenziare le conversioni di tipo \underline{implicite} e scrivere cosa viene stampato a video alla fine, che tipo assume alla fine la variabile \emph{sum} e quanto valgono le variabili \emph{x} e \emph{y} alla fine dell'esecuzione del codice.  


\begin{minipage}[t]{0.5\linewidth}
	\begin{lstlisting}[language=C]
	int  x = 13;
	char y = 'C'; /* ASCII value is 67 */
	float sum;
	sum = x + y;
	printf("sum = %f", sum );
	\end{lstlisting}
\end{minipage}
\begin{minipage}[t]{0.5\linewidth}
	Linea 4, \emph{x} viene convertito da int a float e \emph{y} da char a float. Alla fine viene stampato il valore \emph{80.000000} della variabile \emph{sum} che è di tipo float.
	\\
	Le variabili \emph{x} e \emph{y} alla fine dell'esecuzione del codice, rimangono invariate e quindi assumono valore "13" e "C" rispettivamente.
\end{minipage}



\question[1]
Rappresentare i numeri $1$, $-1$ in complemento a due (8bit) 

\begin{minipage}[t]{0.5\linewidth}
	
\end{minipage}
\begin{minipage}[t]{0.5\linewidth}
	\begin{lstlisting}[language=C]
	1 = 00000001
	-1 = 11111111
	\end{lstlisting}
\end{minipage}



\question[2]
Calcolare la seguente somma in complemento a due e scrivere il risultato sia in complemento a due che in base decimale

\begin{minipage}[t]{0.5\linewidth}
	\begin{lstlisting}[language=C]
	00000101 +
	11110101\end{lstlisting}
\end{minipage}
\begin{minipage}[t]{0.5\linewidth}
	\begin{lstlisting}[language=C]
	00000101 +
	11110101 =
	11111010 = -5
	\end{lstlisting}
\end{minipage}



\question[1]
Scrivere una funzione che dati in \emph{input} due caratteri qualsiasi, stampi a video il carattere e il relativo valore nella codifica ASCII, con valore ASCII più grande (in caso di uguaglianza si stampi il primo). \\
Es. \emph{input}: ("$a$", "$h$"), i loro valori nella codifica ASCII sono: 97 e 104, quindi l' \emph{output} dovrà essere del tipo: "Il valore ASCII del carattere $h$ è $104$".

\begin{minipage}[t]{0.5\linewidth}
	\begin{lstlisting}[language=C]
	void printASCII (char x, char y){
		if (x >= y){
			printf("Il valore ASCII del carattere %c è: %d", x, x);    
		}
		else {
			printf("Il valore ASCII del carattere %c è: %d", y, y);    
		}
	}
	\end{lstlisting}
\end{minipage}



\question[3]
Scrivere una funzione che dati in \emph{input} due numeri interi, calcoli il \emph{massimo comune divisore} di questi.

\begin{minipage}[t]{0.5\linewidth}
	\begin{lstlisting}[language=C]
	int gcd(int a, int b) {
		int gcd = 0;
		for(int i = a; i >= 1; i--) {
			if(a%i == 0 && b%i == 0){ 
				gcd = i;
				break;
			}
		}
		return gcd;
	}
	\end{lstlisting}
\end{minipage}



\question[3]
Scrivere una funzione che richieda di inserire un numero da tastiera e verifichi se questo è \emph{palindromo}, cioè che si legge uguale sia da sinistra verso destra, che da destra verso sinistra (es. 909). Se la condizione è verificata si ritorni il valore intero $1$, altrimenti $0$.


\begin{minipage}[t]{0.5\linewidth}
	\begin{lstlisting}[language=C]
	int palindromo() {
		int a, b, c, s = 0;
		printf("Inserisci un numero:\t");
		scanf("%d", &a);
		c = a;
		while(a > 0){
		  b = a%10;
		  s = (s*10)+b;
		  a = a/10;
	   }
	   
	   if(s == c)
		return 1;
	   else 
		return 0;
	}
	\end{lstlisting}
\end{minipage}



\question[3]
Scrivere un programma che dato un array di interi già definito e la sua dimensione (\emph{input}: int $arr[]$, int $arr\_size$), rimuova i duplicati all'interno di questo e stampi i suoi elementi


\begin{minipage}[t]{\linewidth}
	\begin{lstlisting}[language=C]
	
	for(int i = 0; i < arr_size-1; i++){
        for(int j = i+1; j < arr_size; j++){
            if (arr[i] == arr[j]){
                for (int k = j; k < arr_size-1; k++){
                    arr[k] = arr[k+1];
                }
                j--;
                arr_size--;
            }
        }
    }
    
    for(int i = 0; i < arr_size; i++){
        printf("value%d: %d \n", i+1, arr[i]);
    }
    \end{lstlisting}   
\end{minipage}



\question[2]
Scrivere una funzione cha dato in input un numero intero, stampi a video un triangolo composto da sequenze di numeri incrementali fino al massimo di n e viceversa. Cioè: \\
\begin{minipage}[t]{0.5\linewidth}
	\begin{lstlisting}[language=C]
	Input: 5        Input:3
	1               1
	12              12
	123             123
	1234            12
	12345           1
	1234
	123
	12
	1\end{lstlisting}
\end{minipage}

\begin{minipage}[t]{0.5\linewidth}
	\begin{lstlisting}[language=C]
	void triangolo(int n){
		for(int i=1; i <= n*2; i++){
			if (i <=n){
			   for(int j = 1; j <= i; j++){
				   printf("%d", j);
			   } 
			   printf("\n");
			}
			else{
				for(int j = 1; j <= n*2 - i; j++){
					printf("%d", j);
				}
				printf("\n");
			}
		}
	}
	\end{lstlisting}
\end{minipage}



\question[2]
Scrivere una funzione \underline{iterativa} che, dato in \emph{input} un numero intero, stampi a video il suo valore nella successione di fibonacci

\begin{minipage}[t]{0.5\linewidth}
	\begin{lstlisting}[language=C]
	int fibonacci_iter(int n){
		if (n == 0){
			return 0;
		}
		if (n == 1 || n == 2){
			return 1;
		}
		
		int n1 = 1;
		int n2 = 1;
		int fib;
		for (int i = 2; i <= n; i++){
			fib = n1 + n2;
			n1 = n2;
			n2 = fib;
		}
		return fib;
	}
	\end{lstlisting}
\end{minipage}



\question[2]
Scrivere una funzione \underline{ricorsiva} che dato in \emph{input} un numero intero, stampi a video il suo valore nella successione di fibonacci

\begin{minipage}[t]{0.5\linewidth}
	\begin{lstlisting}[language=C]
	int fibonacci_rec(int n){
		if (n == 0){
			return 0;
		}
		if (n == 1 || n == 2){
			return 1;
		}		
		return fibonacci_iter(n-2) + fibonacci_iter(n-1);
	}
	\end{lstlisting}
\end{minipage}



\question [1]
Scrivere una funzione che, dato in \emph{input} un puntatore (di tipo intero) ed un numero intero, calcoli il fattoriale di questo numero e lo salvi nell'area di memoria indirizzata dal puntatore


\begin{minipage}[t]{0.5\linewidth}
	\begin{lstlisting}[language=C]
	void fattoriale(int *fattoriale, int n){
		if (n == 0){
			*fattoriale = 1;
		}
		else{
			*fattoriale = 1;
			for(int i = 1; i <= n; i++){
				*fattoriale = *fattoriale * i;
			}
		}
	}
	\end{lstlisting}
\end{minipage}



\question[1] 
Definire una struttura per una \emph{linked list} contenente un valore intero e un puntatore al prossimo nodo.


\begin{minipage}[t]{0.5\linewidth}
	\begin{lstlisting}[language=C]
	typedef struct node {
		int val;
		struct node * next;
	} node_t;
	\end{lstlisting}
\end{minipage}



\question[2] 
Scrivere una funzione che, dato in \emph{input} un puntatore alla testa di una linked list (\emph{*head}), iteri e stampi tutti gli elementi della \emph{linked list} sopra definita


\begin{minipage}[t]{0.5\linewidth}
	\begin{lstlisting}[language=C]
	void print_list(node_t * head) {
		node_t * current = head;

		while (current != NULL) {
			printf("%d\n", current->val);
			current = current->next;
		}
	}
	\end{lstlisting}
\end{minipage}



\question[2]
Allocare dinamicamente un'area di memoria 10x10 di tipo double

\begin{minipage}[t]{0.5\linewidth}
	\begin{lstlisting}[language=C]
	double* A = (double*)malloc(10 * 10 * sizeof(double));
	\end{lstlisting}
\end{minipage}



\question[2]
Scrivere cosa stampa a video il seguente codice ed indicare i valori di tipo intero, contenuti all'interno dell'area di memoria indirizzata dal puntatore \emph{ptr}. Inoltre scrivere una porzione di codice che permetta la stampa di tutti gli elementi contenuti nell'area di memoria indirizzata dal puntatore \emph{ptr}.

\begin{minipage}[t]{0.5\linewidth}
	\begin{lstlisting}[language=C]
	#include <stdio.h>
	#include <stdlib.h>

	int main()
	{
		int *ptr;
		ptr = (int *) calloc(5, sizeof(int));
		
		for (int i = 1; i < 5; i++){
			*(ptr+i) = i;
		}
		
		printf("%d", *ptr);
		return 0;
	}
	\end{lstlisting}
\end{minipage}
\begin{minipage}[t]{0.5\linewidth}
	Stampa a video il numero \emph{0}. \\
	Nell'area di memoria indirizzata da \emph{ptr} sono contenuti i valori: \emph{0, 1, 2, 3, 4}
	
	\begin{lstlisting}[language=C]
	for (int i = 1; i < 5; i++){
		    printf("%d \n", *ptr+i);
	}
	\end{lstlisting}
	
\end{minipage}




\end{questions}

\end{document}